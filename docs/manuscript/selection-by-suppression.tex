\documentclass[]{article}
\usepackage{lmodern}
\usepackage{amssymb,amsmath}
\usepackage{ifxetex,ifluatex}
\usepackage{fixltx2e} % provides \textsubscript
\ifnum 0\ifxetex 1\fi\ifluatex 1\fi=0 % if pdftex
  \usepackage[T1]{fontenc}
  \usepackage[utf8]{inputenc}
\else % if luatex or xelatex
  \ifxetex
    \usepackage{mathspec}
  \else
    \usepackage{fontspec}
  \fi
  \defaultfontfeatures{Ligatures=TeX,Scale=MatchLowercase}
\fi
% use upquote if available, for straight quotes in verbatim environments
\IfFileExists{upquote.sty}{\usepackage{upquote}}{}
% use microtype if available
\IfFileExists{microtype.sty}{%
\usepackage{microtype}
\UseMicrotypeSet[protrusion]{basicmath} % disable protrusion for tt fonts
}{}
\usepackage[margin=1in]{geometry}
\usepackage{hyperref}
\hypersetup{unicode=true,
            pdfborder={0 0 0},
            breaklinks=true}
\urlstyle{same}  % don't use monospace font for urls
\usepackage{graphicx,grffile}
\makeatletter
\def\maxwidth{\ifdim\Gin@nat@width>\linewidth\linewidth\else\Gin@nat@width\fi}
\def\maxheight{\ifdim\Gin@nat@height>\textheight\textheight\else\Gin@nat@height\fi}
\makeatother
% Scale images if necessary, so that they will not overflow the page
% margins by default, and it is still possible to overwrite the defaults
% using explicit options in \includegraphics[width, height, ...]{}
\setkeys{Gin}{width=\maxwidth,height=\maxheight,keepaspectratio}
\IfFileExists{parskip.sty}{%
\usepackage{parskip}
}{% else
\setlength{\parindent}{0pt}
\setlength{\parskip}{6pt plus 2pt minus 1pt}
}
\setlength{\emergencystretch}{3em}  % prevent overfull lines
\providecommand{\tightlist}{%
  \setlength{\itemsep}{0pt}\setlength{\parskip}{0pt}}
\setcounter{secnumdepth}{0}
% Redefines (sub)paragraphs to behave more like sections
\ifx\paragraph\undefined\else
\let\oldparagraph\paragraph
\renewcommand{\paragraph}[1]{\oldparagraph{#1}\mbox{}}
\fi
\ifx\subparagraph\undefined\else
\let\oldsubparagraph\subparagraph
\renewcommand{\subparagraph}[1]{\oldsubparagraph{#1}\mbox{}}
\fi

%%% Use protect on footnotes to avoid problems with footnotes in titles
\let\rmarkdownfootnote\footnote%
\def\footnote{\protect\rmarkdownfootnote}

%%% Change title format to be more compact
\usepackage{titling}

% Create subtitle command for use in maketitle
\newcommand{\subtitle}[1]{
  \posttitle{
    \begin{center}\large#1\end{center}
    }
}

\setlength{\droptitle}{-2em}

  \title{}
    \pretitle{\vspace{\droptitle}}
  \posttitle{}
    \author{}
    \preauthor{}\postauthor{}
    \date{}
    \predate{}\postdate{}
  
\usepackage[left]{lineno}
\linenumbers
\usepackage{setspace}
\doublespacing
\DeclareUnicodeCharacter{200E}{}

\begin{document}

\section{Selection by suppression}\label{selection-by-suppression}

Michael J. Koontz\textsuperscript{1,2,*}, Zachary L.
Steel\textsuperscript{3}, Andrew M. Latimer\textsuperscript{1,2},
Malcolm P. North\textsuperscript{1,2,4}

\textsuperscript{1}Graduate Group in Ecology, University of Californa,
Davis, CA, USA\\
\textsuperscript{2}Department of Plant Sciences, University of
California, Davis, CA, USA\\
\textsuperscript{3}Department of Environmental Science and Policy,
University of California, Berkeley, CA, USA\\
\textsuperscript{4}USDA Forest Service, Pacific Southwest Research
Station, Davis, CA, USA

\textsuperscript{*}Correspondence: \texttt{michael.koontz@colorado.edu}

Date report generated: March 12, 2019

\subsection{Abstract}\label{abstract}

\subsection{Introduction}\label{introduction}

Fire suppression is an oft-cited root cause of the modern trend of
larger, more severe wildfires in the yellow pine/mixed-conifer forests
of California's Sierra Nevada mountain range. While this system would
experience frequent, low- to moderate-severity wildfire every 8 to 15
years in the several centuries prior to Euroamerican settlement,
suppression

Bimodal distribution of fire sizes under suppression policy: either the
fires were quickly put out and remained very small, or they escaped
suppression efforts and grew exceptionally large due to regional climate
conspiring with accumulated fuel conditions. Indeed, Miller and Safford
(2017) found evidence for this pattern in that the average size of all
fires is much smaller under a modern fire suppression management regime
compared to pre-Euroamerican settlement fires (as one might intuitively
expect given the goal of suppression is to reduce fire size), but the
average size of larger fires (\textgreater{}4 hectares) is
counter-intuitively much greater under modern suppression management.
Many studies have suggested that adding more fire to the landscape is a
way to return forests to pre-Euroamerican settlement resilient
conditions (North \emph{et al.} 2015) (XXXXX Stephens, Stevens,
Collins). This

Thus, the cumulative effect of fire suppression policy has led to a
management paradox with respect to maintaining forest health: \emph{we
shouldn't put out the fires that we can, but we can't put out the fires
that we should}. Climate change has complicated the paradox further.
Earlier snowmelt, drier and hotter conditions, as well as longer fire
seasons (Westerling 2006, 2016; Abatzoglou and Kolden 2013; Abatzoglou
and Williams 2016).

We use a comprehensive dataset of fire occurrence (Short 2017) as well
as a new dataset of fire severity (Koontz \emph{et al.} 2019a) to
measure this ``selection by suppression'' effect.

\subsection{Methods}\label{methods}

Description of forest type. Using FRID designation of yellow
pine/mixed-conifer to represent ``potential vegetation'' (Harvey
\emph{et al.} 2016; Steel \emph{et al.} 2018; Koontz \emph{et al.}
2019b).

Some summary stats of the Sierra Nevada yellow pine/mixed-conifer subset
of the Short (2017) fire occurrence dataset.

Some summary stats of the Koontz \emph{et al.} (2019a) severity dataset.

\subsection{Results}\label{results}

\subsection{Discussion}\label{discussion}

My plan is to write a paper about the interaction between fire size and
suppression management using the new dataset of severity in YPMC that
includes fires down to 4 hectares in size.

The idea was originally suggested by Jennifer Balch as something she was
curious about while I was interviewing for the Earth Lab postdoc. The
development from there has largely been shaped by how you've talked
about the ``selection'' effect of suppression resulting in especially
large fires because the ones that grow large are only able to do so
because they burn under extreme conditions. Steel \emph{et al.} (2018)
found only a small difference in the measure of regional climate during
suppression versus wildfire use fires (modeled fuel moisture), which led
to the conclusion that an ``extreme fuel'' effect underlied the
differences in the size/configuration of high and unburned severity
patches. But that analysis only included fires greater than 80 hectares
in size, which is still pretty big. Analyzing big fires made sense for
that paper, because they affect the most area, but if we want to suggest
``let fires burn at more moderate weather conditions'' as a mitigation
strategy for a century of making fuel conditions more extreme, then I
think this paper would serve the purpose of measuring the degree to
which good work can be done by a fire (even in extreme fuel conditions)
in milder weather conditions.

Good starting points:

Fire occurrence data: (Short 2017) Papers on interacting effects of
vegetation, regional climate, fire size, and severity patterns: (Cansler
and McKenzie 2014; Harvey \emph{et al.} 2016).

\subsection*{References}\label{references}
\addcontentsline{toc}{subsection}{References}

\hypertarget{refs}{}
\hypertarget{ref-abatzoglou2013a}{}
Abatzoglou JT and Kolden CA. 2013. Relationships between climate and
macroscale area burned in the western United States. \emph{International
Journal of Wildland Fire} \textbf{22}: 1003.

\hypertarget{ref-abatzoglou2016}{}
Abatzoglou JT and Williams AP. 2016. Impact of anthropogenic climate
change on wildfire across western US forests. \emph{Proceedings of the
National Academy of Sciences} \textbf{113}: 11770--5.

\hypertarget{ref-cansler2014}{}
Cansler CA and McKenzie D. 2014. Climate, fire size, and biophysical
setting control fire severity and spatial pattern in the northern
Cascade Range, USA. \emph{Ecological Applications} \textbf{24}:
1037--56.

\hypertarget{ref-harvey2016b}{}
Harvey BJ, Donato DC, and Turner MG. 2016. Drivers and trends in
landscape patterns of stand-replacing fire in forests of the US Northern
Rocky Mountains (1984-2010). \emph{Landscape Ecology} \textbf{31}:
2367--83.

\hypertarget{ref-koontz2019}{}
Koontz MJ, Fick SE, and Werner CM \emph{et al.} 2019a. Wildfire
severity, vegetation characteristics, and regional climate for fires
covering more than 4 hectares burning in yellow pine/mixed-conifer
forests of the Sierra Nevada, California, USA from 1984 to 2017.
\emph{Open Science Framework}.

\hypertarget{ref-koontz2019a}{}
Koontz MJ, North MP, and Werner CM \emph{et al.} 2019b. Local
variability of vegetation structure increases forest resilience to
wildfire. \emph{EcoEvoRxiv}.

\hypertarget{ref-miller2017}{}
Miller JD and Safford HD. 2017. Corroborating Evidence of a
Pre-Euro-American Low- to Moderate-Severity Fire Regime in Yellow
PineMixed Conifer Forests of the Sierra Nevada, California, USA.
\emph{Fire Ecology} \textbf{13}: 58--90.

\hypertarget{ref-north2015}{}
North MP, Stephens SL, and Collins BM \emph{et al.} 2015. Reform forest
fire management. \emph{Science} \textbf{349}: 1280--1.

\hypertarget{ref-short2017}{}
Short KC. 2017. Spatial wildfire occurrence data for the United States,
1992-2015 {[}FPA\_FOD\_20170508{]} (4th Edition).

\hypertarget{ref-steel2018}{}
Steel ZL, Koontz MJ, and Safford HD. 2018. The changing landscape of
wildfire: Burn pattern trends and implications for California's yellow
pine and mixed conifer forests. \emph{Landscape Ecology} \textbf{33}:
1159--76.

\hypertarget{ref-westerling2006}{}
Westerling AL. 2006. Warming and Earlier Spring Increase Western U.S.
Forest Wildfire Activity. \emph{Science} \textbf{313}: 940--3.

\hypertarget{ref-westerling2016}{}
Westerling AL. 2016. Increasing western US forest wildfire activity:
Sensitivity to changes in the timing of spring. \emph{Philosophical
Transactions of the Royal Society B: Biological Sciences} \textbf{371}:
20150178.


\end{document}
